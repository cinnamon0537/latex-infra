\documentclass[11pt,a4paper]{article}

% Encoding & Sprache
\usepackage[T1]{fontenc}
\usepackage[utf8]{inputenc}
\usepackage[ngerman]{babel}

% Layout & Typografie
\usepackage[a4paper,margin=2cm]{geometry}
\usepackage{setspace}
\usepackage{enumitem}
\usepackage{hyperref}
\usepackage{graphicx}
\usepackage{array}
\usepackage{booktabs}
\usepackage{multicol}
\usepackage{longtable}
\usepackage{tabularx}
\usepackage{fancyhdr}
\usepackage{titlesec}
\usepackage{xcolor}

\hypersetup{
  colorlinks=true,
  linkcolor=black,
  urlcolor=blue,
  citecolor=black
}

% Header / Footer
\pagestyle{fancy}
\fancyhf{}
\lhead{Studiengang: HCD}
\chead{HopIn Usability Test (A6A)}
\rhead{Assignment: A6A}
\lfoot{Kevin Forter \& Benjamin Feichtlbauer \& Rohit Birdi \& Muhammad Suhail Edipoglu}
\rfoot{\thepage}

% Absätze & Listen
\setlength{\parskip}{6pt}
\setlength{\parindent}{0pt}
\setlist[itemize]{leftmargin=1.2em}
\setlist[enumerate]{leftmargin=1.4em}

% Makros
\newcommand{\appname}{\textbf{HopIn}}
\newcommand{\minisection}[1]{\vspace{0.4em}\textbf{#1}\quad}

\begin{document}

% Titelseite
\begin{titlepage}
  \centering
  {\Large Human-Centered Design (HCD)}\\[2ex]
  {\large Assignment A6A}\\[10ex]
  {\huge \textbf{HopIn}: Moderierter Usability-Test}\\[3ex]
  {\large Testplan, Aufgaben, Messkonzept und Fragebögen}\\[8ex]

  % --- Bild zentral einfügen ---
  \includegraphics[width=0.4\textwidth]{images/iphone_mockup.png}\\[8ex]

  % --- Tabellenbereich ---
  \begin{tabular}{@{}p{4cm}p{9cm}@{}}
    \textbf{Studiengang:} & HCD \\
    \textbf{Kurs / Studierende:} & Kevin Forter \& Benjamin Feichtlbauer \& Rohit Birdi \& Muhammad Suhail Edipoglu \\
  \end{tabular}

  \vfill
\end{titlepage}

\tableofcontents
\newpage

\section{Beschreibung des getesteten Prototyps}
\appname{} ist eine mobile App / Web-App zur \textbf{temporären Event-Organisation} über \textbf{Gruppen}, die nach dem Anlass \textbf{automatisch archiviert} werden. Nutzerinnen und Nutzer können spontane oder kurzfristige Aktivitäten (z.\,B. Partys, Ausflüge, Sport, Uni-Projekte) gemeinsam planen, kommunizieren und durchführen. Im Unterschied zu dauerhaften Messenger-Gruppen bleibt der Kommunikationsbereich in \appname{} aufgeräumt: Nach Ende des Events verschwindet die Gruppe aus der aktiven Liste und ist im Archiv weiterhin einsehbar.

\minisection{Zentrale Funktionen (aus dem Prototyp):}
\begin{itemize}
  \item \textbf{Event-Gruppe erstellen} (Titel, Datum/Zeit, Ort, Beschreibung).
  \item \textbf{Teilnehmende einladen} (Beitrittslink / QR teilen).
  \item \textbf{Chat \& Organisation} an einem Ort: \textbf{To-Dos}, \textbf{Umfragen/Abstimmungen} (z.\,B. Termin/Ort), Hinweise.
  \item \textbf{Gruppe abschließen/archivieren} nach Event-Ende (automatisch oder manuell).
\end{itemize}

\section{Testziele gemäss Jakob Nielsen}
Wir orientieren uns an \textbf{Nielsens fünf Qualitätskomponenten} (Usability-Ziele) und leiten daraus konkrete Messgrößen ab:
\begin{enumerate}
  \item \textbf{Learnability} – Wie leicht können Erstnutzende die Hauptaufgaben durchführen?
  \item \textbf{Efficiency} – Wie schnell und reibungslos laufen die Kernflows ab?
  \item \textbf{Memorability} – Finden Nutzende nach kurzer Ablenkung schnell wieder hinein?
  \item \textbf{Errors} – Welche Fehler treten auf? Wie schwer sind sie? Wie gut lassen sie sich beheben?
  \item \textbf{Satisfaction} – Wie zufrieden sind die Nutzenden (subjektiv)?
\end{enumerate}

\section{Rahmenbedingungen des Tests}
\begin{itemize}
  \item \textbf{Setting:} Moderierter Lab-Test (im Unterricht). Testende sind Kommilitoninnen/Kommilitonen.
  \item \textbf{Dauer pro Person:} 20 Minuten (Intro, ggf.\ Kurzfragebogen, Aufgaben, Wrap-up, Post-Fragebogen).
  \item \textbf{Rollen:} Moderator/in, Protokollant/in (Notizen, Zeiten), ggf.\ stille Beobachter/innen.
  \item \textbf{Protokoll:} \textit{Think-Aloud}; neutrale Nachfragen, keine führenden Fragen.
  \item \textbf{Aufzeichnung:} Falls möglich, Bildschirmaufnahme für spätere Analyse (nur intern).
\end{itemize}

\section{Aufgaben (Tasks)}
Die Aufgaben sind realistisch, zielbasiert und ergebnisoffen formuliert. Zeitlimits dienen nur der Abbruchdefinition.

\subsection*{T1 – Event-Gruppe anlegen (Kernflow)}
\textbf{Ziel:} Learnability, Efficiency.\\
\textbf{Beschreibung:} ``Stell dir vor, ihr plant für \textbf{morgen} ein gemeinsames \textbf{Sporttreffen} in der Nähe. \textbf{Erstelle eine neue Gruppe}, gib ihr einen passenden Titel, setze \textbf{Ort und Uhrzeit} und füge eine kurze Beschreibung hinzu.'' - (Alles logischerweise hypothetisch, da in FIMGA keine Eingaben gemacht werden können)\\
\textbf{Erfolgskriterien:} Gruppe sichtbar angelegt\\
\textbf{Abbruchkriterium:} 5 Minuten.

\subsection*{T2 – Teilnehmende einladen}
\textbf{Ziel:} Learnability, Errors.\\
\textbf{Beschreibung:} ``\textbf{Lade 2 Freund/innen ein.} Teile dazu den passenden \textbf{Beitrittslink oder QR-Code}.''\\
\textbf{Erfolgskriterien:} Invite-Lösung gefunden, Freigabe initiiert.\\
\textbf{Abbruchkriterium:} 3 Minuten.

\subsection*{T3 – Chat erstellen}
\textbf{Ziel:} Learnability, Memorability.\\
\textbf{Beschreibung:} ``Lege in der Gruppe einen \textbf{Chat} an. Gehe in ein Event mit einem Chat (``Serena's Birthday'') und lese die Konversationen an''\\
\textbf{Erfolgskriterien:} Chat erstellt und bestehende Konversationen durchgelesen.\\
\textbf{Abbruchkriterium:} 4 Minuten.

\subsection*{T4 – To-Do hinzufügen und abhaken}
\textbf{Ziel:} Efficiency, Errors.\\
\textbf{Beschreibung:} ``Erstelle ein \textbf{To-Do} (z.\,B. `Getränke mitbringen') und \textbf{weise es dir selbst zu}. Markiere es anschließend als \textbf{erledigt}.''\\
\textbf{Erfolgskriterien:} To-Do erstellt, zugewiesen und erledigt.\\
\textbf{Abbruchkriterium:} 3 Minuten.

\subsection*{T5 – Gruppe abschließen/archivieren}
\textbf{Ziel:} Memorability, Errors, Satisfaction.\\
\textbf{Beschreibung:} ``Das Event ist vorbei. \textbf{Gehen in das Archiv} und öffne ein archiviertes Event (``z.B. Elternabend'').''\\
\textbf{Erfolgskriterien:} Archiviertes Event geöffnet.\\
\textbf{Abbruchkriterium:} 3 Minuten.

\section{Messkonzept: Was wird gemessen / notiert}
\begin{itemize}
  \item \textbf{Task-Erfolg} (bestanden / nicht bestanden / mit Hilfe).
  \item \textbf{Zeit pro Task} (Start--Ende, in Sekunden).
  \item \textbf{Fehler \& Umgehungen} (Art, Häufigkeit, Schweregrad 0--4).
  \item \textbf{Rückfragen/Hilfen} (Anzahl, Art).
  \item \textbf{Beobachtungen \& Zitate} (Verständnisprobleme, Erwartung vs.\ Systemverhalten).
  \item \textbf{Subjektive Bewertung} (Likert-Skalen; Post-Fragebogen).
\end{itemize}

\subsection*{Schweregradskala (0--4)}
\begin{itemize}
  \item 0 = Kosmetik \quad 1 = Klein \quad 2 = Mittel \quad 3 = Groß \quad 4 = Kritisch (Blocker)
\end{itemize}

\section{Fragebögen}
\subsection*{Kurz-Intake (optional, 1 Minute)}
\begin{itemize}
  \item Erfahrung mit Event-/Gruppen-Apps (z.\,B. WhatsApp-Gruppen, Telegram, Discord): \textit{Nie / Selten / Manchmal / Oft / Sehr oft}
  \item Bevorzugtes Gerät: \textit{Smartphone / Desktop / Beides}
\end{itemize}

\subsection*{Post-Fragebogen (ca.\ 2 Minuten)}
\begin{minipage}{0.9\linewidth}
\textbf{UMUX-Lite} (1--7, 1 = stimme gar nicht zu, 7 = stimme voll zu)
\end{minipage}\\[-0.6em]
\begin{enumerate}
  \item \appname{} \textit{ist nützlich} für das, was ich tun möchte.
  \item \appname{} \textit{ist einfach zu bedienen}.
\end{enumerate}
\textbf{Zusatz (1--7):}\\[-0.6em]
\begin{enumerate}[resume]
  \item Insgesamt bin ich mit \appname{} zufrieden.
  \item Ich würde \appname{} Freund/innen empfehlen.
\end{enumerate}
\begin{itemize}
  \item Was war \textbf{am hilfreichsten}? \_\_\_\_\_\_
  \item Was war \textbf{am nervigsten/unverständlichsten}? \_\_\_\_\_\_
  \item \textbf{Wünsche/Ideen} für Verbesserungen: \_\_\_\_\_\_
\end{itemize}

\section{Ablauf \& Moderationsleitfaden (20 Minuten)}
\begin{enumerate}
  \item \textbf{Begrüßung \& Einverständnis} (2 Min): Zweck, Aufzeichnung, dass \textit{das Produkt im Test ist}.
  \item \textbf{Kurz-Intake} (optional, 1 Min).
  \item \textbf{Aufgaben T1--T5 mit Think-Aloud} (12--14 Min). Neutrale Nachfragen: ``Was erwarten Sie hier?'' --- keine Führung.
  \item \textbf{Wrap-up} (1--2 Min): Top-3 Probleme, größter Aha-Moment.
  \item \textbf{Post-Fragebogen} (2 Min).
\end{enumerate}

\section{Beobachtungsbögen \& Formulare (Druckvorlagen)}
\subsection*{B1 – Einverständniserklärung (Kurz)}
Mit Ihrer Zustimmung zeichnen wir diese Sitzung auf. Die Aufzeichnung dient nur der Auswertung im Kurs und wird nicht extern geteilt. Sie können die Teilnahme jederzeit ohne Nachteile abbrechen.\\[0.4em]
\textbf{Name:} \rule{0.6\linewidth}{0.4pt}\quad \textbf{Datum/Unterschrift:} \rule{0.3\linewidth}{0.4pt}

\subsection*{B2 – Task-Tracking (pro Person)}
\begin{longtable}{p{1.6cm}p{2.6cm}p{3.2cm}p{1.2cm}p{1.2cm}p{5.4cm}}
\toprule
\textbf{Task} & \textbf{Start--Ende (s)} & \textbf{Erfolg (Ja/Nein/Hilfe)} & \textbf{Fehler} & \textbf{Schwere} & \textbf{Beobachtungen \& Zitate}\\
\midrule
T1 & & & & & \\
T2 & & & & & \\
T3 & & & & & \\
T4 & & & & & \\
T5 & & & & & \\
\bottomrule
\end{longtable}

\subsection*{B3 – Post-Fragebogen (UMUX-Lite \& Zusatz)}
\begin{tabularx}{\linewidth}{*{7}{>{\centering\arraybackslash}p{0.9cm}} p{10cm}}
\toprule
\multicolumn{7}{c}{\textbf{Antwortskala (1 = gar nicht, 7 = voll)}} & \\[-0.3em]
\cmidrule(lr){1-7}
\textbf{1} & \textbf{2} & \textbf{3} & \textbf{4} & \textbf{5} & \textbf{6} & \textbf{7} & \textbf{Item}\\
\midrule
 & & & & & & & \appname{} ist nützlich für das, was ich tun möchte.\\
 & & & & & & & \appname{} ist einfach zu bedienen.\\
 & & & & & & & Insgesamt bin ich mit \appname{} zufrieden.\\
 & & & & & & & Ich würde \appname{} Freund/innen empfehlen.\\
\bottomrule
\end{tabularx}

\vspace{1em}
\begin{tabularx}{\linewidth}{|X|}
\hline
Was war am hilfreichsten? \vspace{2cm}\\\hline
Was war am nervigsten/unverständlichsten? \vspace{2cm}\\\hline
Wünsche/Ideen für Verbesserungen: \vspace{2.5cm}\\\hline
\end{tabularx}

\section{Auswertung \& Reporting (Hinweise)}
\begin{itemize}
  \item Kennzahlen je Task: Erfolgsrate, Median-Zeit, Fehlerarten, Hilfen.
  \item Schweregrad-Priorisierung (Impact $\times$ Häufigkeit) der gefundenen Probleme.
  \item Ableitung konkreter Design-/Content-Empfehlungen je Fund (mit Screenshot, falls möglich).
  \item Optional: Gesamt-\textit{UMUX-Lite}-Score (Mittelwert der zwei Kernitems skaliert auf 0--100).
\end{itemize}

\end{document}
