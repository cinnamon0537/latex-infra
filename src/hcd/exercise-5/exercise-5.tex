\documentclass[a4paper,12pt]{article}

% ===== Encoding & Language =====
\usepackage[T1]{fontenc}
\usepackage{lmodern}
\usepackage[utf8]{inputenc}
\usepackage[ngerman]{babel}
\usepackage[tracking=true,final]{microtype}
\usepackage{needspace}

% ===== Layout =====
\usepackage{geometry}
\geometry{margin=1in}
\raggedbottom
\usepackage{parskip}
\usepackage{graphicx}
\graphicspath{{./}{./images/}}
\usepackage[font=small,labelfont=bf]{caption}
\usepackage{float}

% ===== Header / Footer =====
\usepackage{fancyhdr}
\pagestyle{fancy}
\fancyhf{}
% >>> EDIT THESE TWO LINES FOR EACH PROTOCOL <<<
\fancyhead[L]{Kursname – Aufgabentitel}
\fancyfoot[C]{Autor:innen – Seite \thepage}
\setlength{\headheight}{14pt}

% ===== Links & PDF meta =====
\usepackage[hidelinks]{hyperref}
\hypersetup{
  pdftitle={PROTOKOLL – <Thema/Titel>},
  pdfauthor={<Autor:innen>},
  pdfsubject={<Kurs/Lehrveranstaltung>},
  pdfcreator={LaTeX}
}

% ===== Quick macros for repeating patterns =====
\newcommand{\pos}{\textbf{Positiv:} }
\newcommand{\negx}{\textbf{Negativ:} }
\newcommand{\impr}{\textbf{Verbesserungsvorschlag:} }

% Optional: compact itemize
\usepackage{enumitem}
\setlist{itemsep=2pt,topsep=4pt}

\begin{document}

% ===================== TITELBLATT =====================
\begin{titlepage}
  \centering
  \vspace*{2cm}
  {\Huge Protokoll – \textit{<Kursname>} \par}
  \vspace{0.6cm}
  {\Large \textbf{<Aufgabe/Thema>} \par}
  \vspace{1.6cm}
  {\large <Autor:innen> \par}
  {\large <Studiengang / Semester> \par}
  {\large Lehrveranstaltung: <Kursname> \par}
  {\large \today\par}
\end{titlepage}

% ===================== EINLEITUNG =====================
\section*{Einleitung}
Kurz das Ziel, den Kontext und den Umfang dieser Übung/Analyse beschreiben.
Was wird untersucht, welche Methode wird genutzt, und was ist das erwartete Ergebnis?

% ===================== HAUPTTEIL =====================
\section*{Aufgaben / Experimente / Kriterien}
% Duplikatfähiger Block – kopieren für jede Teilaufgabe/Heuristik/Experiment
\subsection*{1. <Teilabschnitt / Kriterium / Aufgabe>}
\pos Kurzer Satz zu einer Stärke oder gutem Befund. \\
\negx Kurzer Satz zu einer Schwäche oder einem Problem. \\
\impr Konkreter, umsetzbarer Vorschlag.

% Beispielbild (optional)
\begin{figure}[H]
  \centering
\IfFileExists{images/example.png}{\includegraphics[width=0.75\textwidth]{images/example.png}}{\fbox{\parbox[c][0.25\textheight][c]{0.75\textwidth}{\centering\small (placeholder: images/example.png not found)}}}
  \caption{Sprechende Bildunterschrift (Was zeigt das Bild? Warum ist es relevant?)}
\end{figure}

\Needspace{0.5\textheight}
\subsection*{2. <Weiteres Kriterium / Aufgabe>}
\pos ...
\negx ...
\impr ...

% Falls du Heuristiken (à la Nielsen) nutzt, ersetze die Untertitel durch „1. Sichtbarkeit ...“, etc.
% Für technische Übungen (Cloud/DevOps) nutze je Abschnitt: Ziel → Schritte → Ergebnis → Screenshots → Stolpersteine.

% ----- Beispiel für technischen Abschnitt -----
\section*{Exercise <Nr.> – <Kurztitel>}
\subsection*{Ziel}
1–2 Sätze: Was soll erreicht werden?

\subsection*{Vorgehen (Kurzprotokoll)}
\begin{enumerate}
  \item Schritt 1 (Kommando, AWS/CLI/Tool, UI-Pfad)
  \item Schritt 2 (wichtige Parameter, Varianten)
  \item Schritt 3 (Validierung/Check)
\end{enumerate}

\subsection*{Ergebnis / Nachweis}
Stichpunkte + ggf. Link/Screenshot (z. B. EC2 Details, Security Group Inbound, Browser Hello-World).

\subsection*{Probleme \& Lösungen (Lessons Learned)}
\begin{itemize}
  \item Problem: \textit{Symptom}. Lösung: \textit{Fix/Workaround}.
  \item …
\end{itemize}

% ===================== FAZIT =====================
\section*{Fazit}
2–4 Sätze: Wichtigste Erkenntnisse, was gut funktionierte, was verbessert werden kann,
und evtl. Ausblick auf nächste Übung/Iteration.

\end{document}
